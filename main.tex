\documentclass[9pt]{article}

\usepackage{titlesec}
\usepackage{titling}
\usepackage[margin=.45in]{geometry}
\usepackage{hyperref}
\usepackage{xcolor}

% Images
\usepackage{graphicx}
\DeclareGraphicsExtensions{.pdf,.png,.jpg,.gif}

\setmainfont{Times New Roman}

\titleformat{\section}{\vspace{-.5em}\Large\bf\raggedright}{}{1em}{}[{\titlerule[2pt]}]
\titlespacing{\section}{0pt}{-3pt}{7pt}

\titleformat{\subsubsection}[runin]
{\bfseries}
{$\bullet$ }
{0em}
{}[:]
\titlespacing{\subsubsection}{0em}{0em}{0.4em}
 
\newcommand{\job}[3]{
    \begin{minipage}[t]{.45\textwidth}
        {\textbf{#1}}

        \textit{#2}
    \end{minipage}
    \begin{minipage}[t]{.5\textwidth}
        \raggedleft{#3}
    \end{minipage}

    \vspace{-.5em}
}

\newcommand{\learn}[5]{
    \begin{minipage}[t]{.2\textwidth}
        #1
    \end{minipage}
    \begin{minipage}[t]{.7\textwidth}
        {\bfseries #2}

        {#3}

        \textit{#4, средний балл: #5}
    \end{minipage}

    \vspace{.5em}
}

\newcommand{\project}[4]{
    \begin{minipage}[t]{.1\textwidth}
        #1
    \end{minipage}
    \begin{minipage}[t]{.87\textwidth}
        {\bfseries #2}

        \textit{#3}

        \vspace{-.8em}

        {#4}
    \end{minipage}
}

\newcommand{\myhref}[3][cyan]{\href{#2}{\color{#1}{#3}}}%

\newcommand{\mailme}{
    \myhref{mailto:hrustbas@gmail.com}{E-Mail}
}
\newcommand{\tg}{
    \myhref{https://t.me/wtukatyr}{Telegram}
}
\newcommand{\linkedin}{
    \myhref{https://www.linkedin.com/in/rustam-basyrov-978b78286/}{LinkedIn}
}
\newcommand{\github}{
    \myhref{https://github.com/rustbas}{GitHub}
}
\newcommand{\phone}{
    {\myhref{tel:+79207902655}{+7-920-790-26-55}}
}

\author{Басыров Рустам}
\date{\today}

\renewcommand{\maketitle}{
    \begin{center}
        {\huge\bfseries\theauthor}

        \Large \mailme | \tg | \phone | \linkedin | \github
    \end{center}

    \vspace{-.2em}
}


\begin{document}

\maketitle

\section{ПРОФЕССИОНАЛЬНЫЙ ОПЫТ}

    \job
        {ООО МОРОЗКО}
        {Системный администратор}
        {Август 2023 года --- по настоящее время}
        {
            \begin{itemize}
                \setlength\itemsep{-.5em}
                \item Поддержка и настройка контейнеров Docker (в командной строке и с помощью Portainer).
                \item Разработка ansible-плейбука для развертывания агента Zabbix на платах Raspberry Pi и последующего мониторинга подключения ИБП к сети питания.
                \item Установка и настройка сертификатов для работы пользователей в онлайн-сервисах.
                \item Поддержка Zabbix — настройка дашбордов, добавление новых хостов.
                \item Поддержка почтового сервера на базе MDaemon — управление учетными записями, устранение неполадок.
                \item Установка и настройка АРМ.
                \item Тех. поддержка пользователей.
            \end{itemize}
        }
    
    \job
        {Институт биоинформатики}
        {Ассистент преподавателя по курсу “Командная строка Linux”}
        {Сентябрь 2023 года — Декабрь 2023 года}
        {
            \begin{itemize}
                \setlength\itemsep{-.5em}
                \item Актуализация курса для учёбы в MacOS.
                \item Ответы на вопросы студентов.
            \end{itemize}
        }

\section{НАВЫКИ}

    \subsubsection{Администрирование}
    дистрибутивы Linux, права доступа, управление процессами, основы docker и ansible, nginx (веб-сервер),
    сетевые протоколы.
    
    \subsubsection{Программирование}
    python (в том числе: numpy, scipy, pandas, matplotlib), bash, C/C++.
    
    \subsubsection{Общее}
    чтение документации (на русском и английском), git, терминал (vim, tmux, различные утилиты командной строки).
    
    \vspace{.5em}

\section{ОБРАЗОВАНИЕ}

    \learn
    {2022 --- 2023}
    {Институт биоинформатики}
    {Заочная программа профессиональной переподготовки}
    {«Алгоритмическая биоинформатика»}
    {4.3/5}
    
    \learn
    {2017 — 2022}
    {Московский Авиационный Институт}
    {Специалитет}
    {«Системы управления летательных аппаратов»}
    {4.73/5}

\section{ДОПОЛНИТЕЛЬНЫЕ КУРСЫ}

    \begin{itemize}
        \setlength\itemsep{-.5em}
        \item Введение в Linux (Институт биоинформатики); stepik.org.
        \item Python. Основы и применение (Институт биоинформатики); stepik.org.
        \item «Докер с нуля»; karpov.courses.
        \item А также \myhref{https://drive.google.com/drive/folders/1EPNQ5b6PDVWkMzCplopYXnyJJu4l9fev?usp=drive_link}{другие}.
    \end{itemize}
    
    \vspace{-.4em}

\section{ПРОЕКТЫ}

    \project
    {2022}
    {Институт Биоинформатики — научный проект}
    {Analysis of the structural diversity of β-arches (научный проект ИБ), (репозиторий/elibrary)}
    {
        \begin{itemize}
            \setlength\itemsep{-.5em}
            \item Рассчитал попарные среднеквадратическые отклонения β-арок и кластеризовал их методом иерархической кластеризации.
            \item Уменьшил количество измерений с 10 до 3 для визуализации распределения β-арок по торсионным углам методом главных компонент.
            \item Проверил гипотезу о равенстве медиан торсионных углов у разных типов β-арок с помощью T-критерия Вилкоксона.
        \end{itemize}
    }
    
    \project
    {2022}
    {Московский Авиационный Институт — дипломная работа}
    {Алгоритм оптимизации траектории полета при свободной маршрутизации}
    {
        \begin{itemize}
            \setlength\itemsep{-.5em}
            \item Разработал алгоритм построения графа по координатам ППМ с учетом запретных зон.
            \item Решил задачу обхода запретной зоны.
            \item Разработал ПО для оптимизации и визуализации траектории полета при свободной и несвободной маршрутизации.
        \end{itemize}
    }
    
    \project
    {2019}
    {Московский Авиационный Институт — научная работа}
    {“Исследование влияния погрешностей БИНС на точность адаптивного управления движением поврежденного самолета” (elibrary)}
    {
        \begin{itemize}
            \setlength\itemsep{-.5em}
            \item Занимался сбором данных влияния шумов датчиков на результат моделирования системы управления.
        \end{itemize}
    }


\section{ОБЛАСТИ ИНТЕРЕСОВ}

    Люблю Linux и все, что связано с ним. Предпочитаю командную строку графическим приложениям.
    Также увлекаюсь математической статистикой и машинным обучением.
    В свободное время, если есть возможность, занимаюсь спортивным туризмом, в частности, горным. В 2019 году
    взошел на Казбек (5033 м, 2Б к.с.) и в 2021 году — на Эльбрус (5642 м, 2А к.с.).

\end{document}
