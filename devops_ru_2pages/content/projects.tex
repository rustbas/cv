\section{ПРОЕКТЫ}

    \project
    {2022}
    {Институт Биоинформатики — научный проект}
    {Analysis of the structural diversity of $\beta$-arches, 
    (\myhref{https://bitbucket.org/stanislavspbgu/fibrils_3d/src/master/}{репозиторий}/%
    \myhref{https://www.elibrary.ru/item.asp?id=49878530}{elibrary})}
    {
        \begin{itemize}
            \setlength\itemsep{-.5em}
            \item Рассчитал попарные среднеквадратическые отклонения $\beta$-арок и кластеризовал их методом иерархической кластеризации.
            \item Уменьшил количество измерений с 10 до 3 для визуализации распределения $\beta$-арок по торсионным углам методом главных компонент.
            \item Проверил гипотезу о равенстве медиан торсионных углов у разных типов $\beta$-арок с помощью T-критерия Вилкоксона.
        \end{itemize}
    }
    
    \project
    {2022}
    {Московский Авиационный Институт — дипломная работа}
    {Алгоритм оптимизации траектории полета при свободной маршрутизации}
    {
        \begin{itemize}
            \setlength\itemsep{-.5em}
            \item Разработал алгоритм построения графа по координатам ППМ (промежуточных пунктов маршрута) с учетом запретных зон.
            \item Решил задачу обхода запретной зоны.
            \item Разработал ПО для оптимизации и визуализации траектории полета при свободной и несвободной маршрутизации.
        \end{itemize}
    }
    
    \project
    {2019}
    {Московский Авиационный Институт — научная работа}
    {“Исследование влияния погрешностей БИНС (безинерциальной навигационной системы) на точность адаптивного управления движением поврежденного самолета” 
    (\myhref{https://elibrary.ru/item.asp?id=39284233&pff=1}{elibrary})}
    {
        \begin{itemize}
            \setlength\itemsep{-.5em}
            \item Занимался сбором данных влияния шумов датчиков на результат моделирования системы управления.
        \end{itemize}
    }
